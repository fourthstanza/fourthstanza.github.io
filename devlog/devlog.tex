\documentclass{article}
\usepackage{graphicx} % Required for inserting images
\usepackage{hyperref}

\title{Github Portfolio Website Development Log}
\author{Matthew Sylvester}
\date{January 2026 -- current}

\begin{document}

\maketitle

\section{Scaffolding}

React + vite with tailwindcss for styling. See tutorial on setup \& basic page structuring followed \href{https://www.youtube.com/watch?v=cIYdiRDFWQw}{here.}

\subsection{React requirements}

Installing requirements and setting up the React development workspace

\begin{itemize}
    \item \textbf{Node.js}; Javascript runtime. Installed from \href{https://nodejs.org/en/download}{https://nodejs.org/en/download}
    \item \textbf{npm CLI}; package manager for Javascript, installed automatically with node.js but releases will not necessarily automatically update. 
    \item \textbf{Vite}; build tool for web projects (web server \& build command which bundles code with Rollup). Installed through npm CLI. Documentation: \href{https://vite.dev/guide/}{https://vite.dev/guide/}
\end{itemize}

Using typescript as the default language for all pages, be aware that the tutorials I've followed are often for javascript/jsx so I may not be properly using typescript. default file extension is tsx unless required to be otherwise.\\


Begin by installing the Node.js runtime. Make sure npm package manager is included in the installer. Once Node.js is installed, open a terminal in the workspace for the project in the code editor and run 

\begin{verbatim}
    npm create vite@latest my-app -- --template react
\end{verbatim}

Note that this command needs to be run only to set up vite and create the index.html file.\\

For the website, the options selected were

\begin{enumerate}
    \item use rolldown-vite?: yes
    \item hosted on \href{http://localhost:5173/}{http://localhost:5173/}
\end{enumerate}

use --host to expose.

When opening the project for a development session, you will need to start the dev server with the following command:
\begin{verbatim}
    npm run dev
\end{verbatim}

\subsection{Styling}

\href{https://lucide.dev/}{Lucide-react} is installed for icons.

\section{Deployment}

Deployed as a github pages page in \href{https://github.com/fourthstanza/fourthstanza.github.io}{fourthstanza.github.io} repo using \href{https://vite.dev/guide/static-deploy.html}{default vite github pages workflow}.

\section{Structure}

\subsection{Navigation bar}

Separated into its own component in \textit{layout/navbar.tsx} for reusability across pages. Contains links to main sections of the site.\\

Has split styling for desktop and mobile views. Mobile view is a hamburger menu that expands to show links when clicked. See use of md: for handling switching between desktop and mobile views.\\

Animated with CSS styling for fade-in and fade-out when opening and closing the mobile menu. See utilities layer.

\subsection{EN/FR button}

Manual implementation of language switching button in the navbar. Currently non-functional. Will be implemented by switching between two versions of each page.

\subsection{Main pages}

Site is separated into sections via the \textit{sections} folder. The sections are as follows.

\begin{itemize}
    \item \textbf{about}; the index and primary page. Contains information about me and a summary of my experience/projects
    \item \textbf{projects}; summary of projects I'm currently working on. Contains additional project pages that go into further detail, links to other resources
    \item \textbf{gallery}; gallery of pictures I've taken
    \item \textbf{contact}; contact \& location information (pull template \& framework from somewhere?)
\end{itemize}

\subsection{Footer}

Basic footer element contains blurb, duplicate nav, and social links.

\section{Routing}

To be set up with react-router-dom for multi-page functionality.

\end{document}
